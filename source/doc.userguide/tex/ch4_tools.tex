%
% VBrowser
%

\chapter{Other Tools}
\label{chap:other_tools}

\section{Application and Tools} 

Other applications currently available in the VL-e Toolkit are:

\begin{itemize}
  \item \Bold{GridProxyDialog}: for creating and managing Grid Proxies. 
  \item \Bold{VLTerm}:		    simple vt100 emulator with some
  xterm extensions.  
  \item \Bold{uricopy.sh,urils.sh}:		URI copy and list script. 
  \item \Bold{jython.sh}:		Jython startup script. 
\end{itemize}

\section{GUI utils}

\subsection{GridProxyDialog}

Some part of the \vbrowser\ can be used stand alone. 
The Grid Proxy Init dialog can be called by running the
\Path{GridProxyDialog.jar} as follows:\\

	\tab \Code{java -jar \$VLET\_INSTALL/bin/GridProxyDialog.jar}\\

Or double click on the jar file in the \Path{VLET\_INSTALL/bin} directory:\\

	\tab \Path{VLET\_INSTALL/bin/GridProxyDialog.jar}\\
	
You cannot move this file out of the installation, but you can create a shortcut
to the jar file instead. 

\subsection{VLTerm}
\label{section:vlterm}

The VL-e Toolkit has an beta version of a VT100 terminal emulation program.
This terminal emulator can be used as a backup if there is no standard xterm or SSH
terminal program available on the user's host.\\
Most VT100 and some XTERM control codes are supported. For basic remote
command execution this terminal application can be used.\\
To start the \Code{VLTerm} application use the command line as follows:\\

	\tab \Code{java -jar \$VLET\_INSTALL/bin/vlterm.jar}\\

Or double click on the jar file at the following path:\\

	\tab \Path{VLET\_INSTALL/bin/vlterm.jar}\\

You cannot move this file out of the installation, but you can create a
shortcut to the jar file instead. \\ 

 \begin{figure}[htbp]
 \centerline{\includegraphics[scale=0.75]{vlterm}}
 \caption{VLTerm at SARA}
 \label{fig:vlterm}
 \end{figure}

\section{Command line tools}

\subsection{URI copy script: uricopy.sh}

At this moment there is only one script interface to the Virtual Resource System
(VRS). This script is the URI copy script: \Path{uricopy.sh}\\

The location is: \Path{VLETINSTALL/bin/uricopy.sh} and the syntax is as
follows: 

\tab \Path{uricopy.sh [options] \lt source URI\gt\ \lt destination URI\gt\
	[-D\lt PROPERTY\gt =\lt VALUE \gt]*}

Command line argument are: 
\begin{itemize}
  \item \Path{  \lt source URI\gt\ } : source file or directory.
  \item \Path{  \lt dest URI\gt\ } : parent directory to copy to. The source
        file  or directory is always created as child of this parent
        directory.
  \item \Path{ [options] } can be: 

\hspace*{10mm}\begin{minipage}{170mm}
\begin{verbatim}
 -r,-dir    ; perform a recursive copy of the source location (for copying 
              directories). 
 -v [-v]    ; be verbose. provide option twice to be even more verbose. 
 -f, -force ; overwrite (optional) existing target location. 
 -move      ; move resource and delete source URI after copy command.
 -mkdirs    ; create target directory path 
 -result    ; print resulting destination URI as follows: "result=...".
 -debug     ; enable debug output.
\end{verbatim}
\end{minipage}
\end{itemize}

The properties which can be specified are depending on the source and
destination URIs. Most VLET settings can be specified as command line options 
using the \Code{-D\lt NAME\gt =\lt VALUE\gt} syntax. \\
\\
An overview of relevant properties are:\\
\\
\EmphBold{Global properties}:\\
\\
\tab \Code{-DpassiveMode=true} always use passive mode for all file transfers.\\
\\ 
\EmphBold{General usage:}\\
\\
The source and destination URIs are mandatory. Options can be both specified
before and after the URIs. The source URI must be an existing file or directory
and the target URI \Emph{must} be an existing directory! This is to
avoid ambiguity between destination files and directories. The new file or
directory will always be created as a child entry in the target directory.\\
\\
\EmphBold{Copying directories:}\\
\\
To copy a directory specify the \Code{-r} option (for recursive copy).\\
The uricopy command will create the new destination directory which
always will be a child (subdirectory) of the destination URI.\\
\\
\EmphBold{Moving files or directories:}\\
\\
To move a file or a directory specify the \Code{-move} option. The source will
be deleted after and \Emph{only} after a successful copy. \\
\\

\subsection{Examples} 

\subsubsection{LFC uricopy example}

Below an example how to use uricopy to upload a file to an LFC server. 

\hspace*{10mm}\begin{minipage}{170mm}
\begin{verbatim}
$VLET_INSTALL/bin/uricopy.sh file:/home/ptdeboer/hello.txt \
	lfn://lfc.grid.sara.nl/grid/pvier/piter
\end{verbatim}
\end{minipage}

Options or server settings can be provided as either a Java property or a URI
attribute. \\
For example to specify the preferred Storage Elements provide the
\Emph{lfc.listPreferredSEs} option to the uricopy script:\\

\hspace*{10mm}\begin{minipage}{170mm}
\begin{verbatim}
$VLET_INSTALL/bin/uricopy.sh file:/home/ptdeboer/hello.txt \
	lfn://lfc.grid.sara.nl/grid/pvier/piter\
	-Dlfc.listPreferredSEs=srm.grid.sara.nl,tbn18.nikhef.nl
\end{verbatim}
\end{minipage}

The above command will try to register the file at the first Storage Element in
the list, if that fail it will try the second storage element. 

Optional LFC properties can be: \\

\hspace*{10mm}\begin{minipage}{170mm}
\begin{verbatim}
-Dlfc.listPreferredSEs=SEHost1,SEHost2 ; list of preferred Storage Elements
-Dlfc.replicaNrOfTries=5               ; number of retries when reading/creating
                                         a replica. 
-Dlfc.replicaCreationMode=Preferred    ; one of: Preferred, PreferredRandom,
                                         DefaultVO, DefaultVORandom 
-Dlfc.replicaSelectionMode=Preferred   ; one of: Preferred,PreferredRandom,
                                         AllSequential, AllRandom
\end{verbatim}
\end{minipage}\\

% \subsubsection{SRB uricopy example}
% 
% An example how to upload files to the SRB server goes as follows:
% 
% \hspace*{10mm}\begin{minipage}{170mm}
% \begin{verbatim}
% $VLET_INSTALL/bin/uricopy.sh file:/home/ptdeboer/hello.txt \
%    srb://srb.grid.sara.nl:50000/VLENL/home/piter.de.boer.vlenl \
%    -Dsrb.defaultResource=vleGridStore \
%    -Dsrb.username=piter.de.boer \
%    -Dsrb.msdasDomainName=vlenl
% \end{verbatim}
% \end{minipage}
% 
% Optional SRB properties can be: \\
% 
% \hspace*{10mm}\begin{minipage}{170mm}
% \begin{verbatim}
% -Dsrb.username=USERNAME               ; username for the SRB server.
% -Dsrb.defaultResource=vleMatrixStore  ; specify resource to store new file/directories.
% -Dsrb.mdasCollectionHome=/VLENL/home  ; specify default home.
% -Dsrb.mdasDomainName=vlenl            ; specify DomainName.
% -Dsrb.hostname=srb.grid.sara.nl       ; hostname of SRB server.
% -Dsrb.port=50000                      ; port of SRB server.
% \end{verbatim}
% \end{minipage}\\
% 
% If you specify any of the above as command line options, you can omit them in the
% URI. \\
% The option:\\
% \\
% \tab \Code{-Dsrb.defaultResource=vleGridStore}\\
% \\
% may be omitted if the default resource of the SRB server
% (\Code{(srb.defaultResource}) equals the one specified in srb settings file:
% \Path{VLETINSTALL/etc/srbsettings.prop.}\\
% If this option is not or incorrectly set the SRB server might return an error as
% follows:\\
% \\
% \hspace*{10mm}\begin{minipage}{170mm}
% \begin{verbatim}
% 	OBJ_ERR_RES_NOT_REG resource has not been registered -2400
% \end{verbatim}
% \end{minipage}
% 
% This means the SRB server could not 'register' the file at the default 'storage'
% location.\\
% \\
% \EmphBold{SRB upload example using URI attributes}:\\
% \\
% Used URIs can have (URI) attributes which allow Server Properties and Settings
% to be specified as part of the URI.\\
% Check the implementation details, or see in the appendices, which property is
% supported as URI attribute.\\
% An example is as follows:\\
% \\
% \hspace*{10mm}\begin{minipage}{170mm}
% \begin{verbatim}
% $VLET_INSTALL/bin/uricopy.sh file:///home/ptdeboer/hello.txt
%    srb://srb.grid.sara.nl:50000/home/piter.de.boer.vlenl\
%    ?srb.defaultResource=vleGridStore\
%    &srb.username=piter.de.boer\
%    &srb.mdasDomanName=vlenl
% \end{verbatim}
% \end{minipage}\\
% \\ 
% \\
% The difference is that the defaultResource property is now specified as URI
% attribute using the \path{?srb.defaultResource=vleGridStore} syntax. \\
% \\
\subsection{Customized uricopy script}

You can use the uricopy.sh as an example and modify it to your liking. Make sure 
the \VLETINSTALL environment variable is set so the script can find the
installation as follows: \\
\\
\hspace*{5mm}\begin{boxedminipage}{165mm}
\begin{verbatim}

#!/bin/bash 
##
# File   : Example modified uricopy.sh script 
# Version: VLET 1.1
# --- 

# Default installation: 
export VLET_INSTALL=/opt/vlet
export VLET_SYSCONFDIR=$VLET_INSTALL/etc
# Configuration files on PoC R3: 
#export VLET_SYSCONFDIR=/etc/vlet

# set base directory 
BASE_DIR=$VLET_INSTALL

# my options: 
OPTIONS="-DpassiveMode=true -Dsrb.username=piter.de.boer -r -force"

# source vlet settings: 
source $VLET_SYSCONFDIR/vletenv.sh 

##
# VLET java class to start: VFSCopy 
CLASS=nl.uva.vlet.vfs.VFSCopy

# JVM options: 
JVMOPTS="-Dvlet.install.sysconfdir=$VLET_SYSCONFDIR -jar $BASE_DIR/bin/bootstrapper.jar"

# Start bootstrapper which does the rest 
echo "Command:" $JAVA $JVMOPTS $CLASS $OPTIONS $@
$JAVA $JVMOPTS $CLASS $OPTIONS $@
# keep return value: 
RETVAL=$? 

#return exit code from VFSCopy
exit $RETVAL; 

\end{verbatim}
\end{boxedminipage}

\section{Jython}

\subsection{Jython start script: jython.sh}

Since Jython is a pure java implementation of Python, Jython can be used 
to access to full VLET API. To run  a Jython script, call the Jython 
bootstrap script from the VLET installation. \\
The location of this bootstrap script is: \Path{VLETINSTALL/bin/jython.sh} and
the syntax is as follows: \\
\\
\tab \Path{jython.sh [-i] \lt jython file\gt\ }\\
\\
Where option \Path{-i} starts the Jython interpreter in interactive mode (leave
out the file when starting the Jython interpreter in this mode). \\
Jython examples can be seen in: \Path{VLETINSTALL/py}. 
\\
The documentation for the VLET API can be found at: 
\Path{VLETINSTALL/doc/api/index-all.html}. \\ 
\\ 
Below an example how to use the VFS interface from VLET in a Jython 
script:

\hspace*{5mm}\begin{boxedminipage}{165mm}
\begin{verbatim}
#!/opt/vlet/bin/jython.sh
##
# VFSClient jython interface example
# (C) www.vl-e.nl 
##
# File         : vfsclient.py 
# Vlet version : 1.1
# Author       : Piter T. de Boer 
#
# Info :  
#    VLET vfs jython interface example.
#    To start execute this jython file, use:
#        $VLET_INSTALL/bin/jython.sh <jython.file> 
##
import sys
#import VRS objects + VFS Client: 
from nl.uva.vlet.vrl import VRL
from nl.uva.vlet.vfs import VFile,VDir,VFSClient

### Helper methods
def boolstr(value):
  if value == True: 
     return "true" 
  else:
     return "false" 
     
# creat new VFSClient
vfs=VFSClient()

# get Virtual File object: 
dir=vfs.getDir("lfn://lfc.grid.sara.nl:5010/grid/pvier/"); 
print "Remote LFC directory = "+dir.toString()
print "--- Directory Attributes ---" 
print " Directory modification time  =",dir.getModificationTime()
print " Directory access permissions = \""+dir.getPermissionsString()+"\"" 

#get contents of remote Directory 
contents=dir.list()

# array acces: 
print "First file = ",contents [0];

# define print function: 
def PrintNode(prefix,node):
  print prefix+"["+node.getType()+"] "+node.toString();

# apply MyPrint on contents array
print "--- contents of remote directory ---"
[PrintNode(" - ", file) for file in contents]
print "--- ---"

# get local home 
home=vfs.getDir("file:///~"); 
print "Local home=",home; 
\end{verbatim}
\end{boxedminipage}


\hspace*{5mm}\begin{boxedminipage}{165mm}
\begin{verbatim}
### remote navigation 
# get remote file:
fileVRL="lfn://lfc.grid.sara.nl:5010/grid/pvier/piter/test.txt"; 
vrlObject=VRL(fileVRL) 

# dissect VRL (=URI)
print "-- VRL object ---"
print " VRL String    =",fileVRL; 
print " VRL Object    =",vrlObject; 
print " VRL Hostname  =",vrlObject.getHostname(); 
print " VRL Port      =",vrlObject.getPort()
print " VRL Path      =",vrlObject.getPath(); 
print " VRL Extension =",vrlObject.getExtension(); 
print "---"

if (vfs.existsFile(fileVRL) is True):
  print "File exists:"+fileVRL; 
else:
  print "*** Error: Remote file does not exists:"+fileVRL; 

# get/create Virtual File object of remote file
remoteFile=vfs.getFile(fileVRL); 
print "--- File Attributes ---" 
print " modification time  =",remoteFile.getModificationTime()
print " length             =",remoteFile.getLength()
print " access permissions = \""+remoteFile.getPermissionsString()+"\"" 
print " isReadable         =",boolstr(remoteFile.isReadable()) 
print " isWritable         =",boolstr(remoteFile.isWritable()) 
print " mimetype           =",remoteFile.getMimeType()

#get contents of remote Directory 

# getContents() returns byte array, getContentsAsString returns String object 
text=remoteFile.getContentsAsString(); 
print "Contents of remote file = "+text; 

# Navigating example:  "cd .."  
remoteParent=remoteFile.getParent(); 
print "Remote parent = ", remoteParent; 

# example "VDir.hasFile()" (hasDir/hasChild)
if (remoteParent.hasFile(remoteFile.getBasename()) is True): 
   print "Remote parent reports that is has the remote file as child" 
else:
   print "***Error: Parent directory of remote file reports is doesn't have the Child!" 

# copy to local home directory, overwrite existing: 
resultFile=remoteFile.copyTo(home);
print "Copied remote file to:",resultFile; 
print "Check whether file is local:",resultFile.isLocal();
print "Local path of file is:",resultFile.getPath(); 
\end{verbatim}
\end{boxedminipage}


\hspace*{5mm}\begin{boxedminipage}{165mm}
\begin{verbatim}
# get system temp dir: 
print "tempdir'=",vfs.getTempDir(); 

# create Unique Temp Dir() on local home:
tmpdir=vfs.createUniqueTempDir(); 
print "unique tempdir =",tmpdir;

print "end."
sys.exit(0); 
\end{verbatim}
\end{boxedminipage}

See the examples directory \path{VLET_INSTALL/py/*} for more jython scripts. 
